\input article

\parskip5mm
\parindent5mm

\itskip-3mm

\let\linkColor\cBlue

Dear Editor,

thank you for considering this manuscript for publication in Acta Physica Polonica B.

Let me clarify briefly on the slightly non-standard history of this manuscript. In January 2020, its first version was published on arXiv and submitted to EPJC. The submission was refused in a referee process which I found strange, e.g.~it involved a single referee despite my objections and arguments. Now (January 2021), encouraged by my colleagues, I am making another attempt to publish this work. I have amended the original version to include reactions to articles published meanwhile, particularly Ref.~[1]. I find that this publication contains several valuable points, but also several misleading and wrong ones. Let me expand on the most likely only concrete criticisms which could possibly indicate a flaw in my calculations: the estimate of the integration bound $b_{\rm max}$. The corresponding paragraph in the manuscript has been extended, giving more details of my argumentation. Beyond this, I have made an explicit check of my somewhat general argumentation, see Figure 1. As argued in the manuscript text, one can observe that:
\> the hadronic eikonal dominates at low $b$, but for $b \gs 40\un{GeV^{-1}}$ it is negligible wrt.~the Coulomb component,
\> at low $b$, there is a noticeable difference between $\de^{\rm C}$ and $\de^{\rm C}_{\rm asym}$ due to the presence of the electromagnetic form-factors, however for $b \gs 15\un{GeV^{-1}}$ the difference essentially vanishes.

\vskip-3mm
\noindent In conclusion, I am unable to find a mistake in my argumentation and calculation.

With best wishes,

Jan Ka\v spar.

\fig[13cm]{fig/delta_cmp_C_H_asympt.pdf}{de c}{Comparison of various contributions to the complete eikonal $\de^{\rm C+H}$. $\de^{\rm C}$ represents the eikonal due to the Coulomb interaction, $\de^{\rm H}$ due to the hadronic one. This example calculation was performed with the form-factor by Puckett et al.~and $\la = 10^{-3}\un{GeV}$. The two rows represent the two hadronic amplitues considered. The two columns show the same plot, only using different ranges and scales.}


\noindent [1] V.~A.~Petrov, ``Twice More On Coulomb-Nuclear Interference'', arXiv:2003.07654v1 [hep-ph] (2020)

\bye
